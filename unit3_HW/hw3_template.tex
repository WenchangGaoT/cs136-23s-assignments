\documentclass[10pt]{article}

%%% Doc layout
\usepackage{fullpage} 
\usepackage{booktabs}       % professional-quality tables
\usepackage{microtype}      % microtypography
\usepackage{parskip}
\usepackage{times}

%% Hyperlinks always black, no weird boxes
\usepackage[hyphens]{url}
\usepackage[colorlinks=true,allcolors=black,pdfborder={0 0 0}]{hyperref}

%%% Math typesetting
\usepackage{amsmath,amssymb}
\usepackage{pythonhighlight} 
%%% Write out problem statements in blue, solutions in black
\usepackage{xcolor}
\newcommand{\officialdirections}[1]{{\color{purple} #1}}

%%% Avoid automatic section numbers (we'll provide our own)
\setcounter{secnumdepth}{0}

%% --------------
%% Header
%% --------------
\usepackage{fancyhdr}
\fancyhf{}
\fancyhead[C]{\ifnum\value{page}=1 CS 136 - 2023s - HW3 Submission \else \fi}
\fancyfoot[C]{\thepage} % page number
\renewcommand\headrulewidth{0pt}
\pagestyle{fancy}


%% --------------
%% Begin Document
%% --------------
\begin{document}

~~\\ %% add vertical space

{\Large{\bf Student Name: TODO}}

\Large{\bf Collaboration Statement:}

Total hours spent: TODO

I discussed ideas with these individuals:
\begin{itemize}
\item TODO
\item TODO
\item $\ldots$	
\end{itemize}

I consulted the following resources:
\begin{itemize}
\item TODO
\item TODO
\item $\ldots$	
\end{itemize}
~~\\
By submitting this assignment, I affirm this is my own original work that abides by the course collaboration policy.
~~\\
~~\\
Links: 
\href{https://www.cs.tufts.edu/cs/136/2023s/hw3.html}{[HW3 instructions]} 
\href{https://www.cs.tufts.edu/cs/136/2023s/index.html#collaboration}{[collab. policy]} 

\tableofcontents

\newpage

\officialdirections{
\subsection*{1a: Problem Statement}
Define $\Sigma = L L^T$. Show the following:
\begin{align}
| \det (L^{-1}) | = \frac{1}{(\text{det} \Sigma)^{\frac{1}{2}}}
\end{align}
}

\subsection{1a: Solution}
TODO YOUR SOLUTION HERE


\officialdirections{
\subsection*{1b: Problem Statement}
Show that the pdf of $x$ is given by:

\begin{align}
p(x) = \frac{1}{(2\pi)^{\frac{D}{2}}}
\frac{1}{(\text{det} \Sigma)^{\frac{1}{2}}}
e^{-\frac{1}{2} (x-\mu)^T \Sigma^{-1} (x-\mu)}
\end{align}
}

\subsection{1b: Solution}
TODO YOUR SOLUTION HERE

\newpage 
\officialdirections{
\subsection*{1c: Problem Statement}
Complete the Python code below, to show how to turn samples from a standard 1D Gaussian, via NumPy's `randn()` into a sample from a multivariate Gaussian.x}

\subsection{1c: Solution}
\begin{python}
import numpy as np

def sample_from_mv_gaussian(mu_D, Sigma_DD, random_state=np.random):
	''' Draw sample from multivariate Gaussian

	Args
	----
	mu_D : 1D array, size D
		Mean vector
	Sigma_DD : 2D array, shape (D, D)
		Covariance matrix. Must be symmetric and positive definite.

	Returns
	-------
	x_D : 1D array, size D
		Sampled value of Gaussian with provided mean and covariance
	'''
	D = mu_D.size
	L_DD = np.linalg.cholesky(Sigma_DD) # compute L from Sigma
	# GOAL: draw each entry of u_D from standard Gaussian 
	u_D = np.zeros(D) # TODO FIXME use random_state.randn(...)
	# GOAL: Want x_D ~ Gaussian(mean = m_D, covar=Sigma_DD)
	x_D = np.zeros(D) # TODO FIXME transform u_D into x_D 
	return x_D
	return x_D
\end{python}



\officialdirections{
\subsection*{2a: Problem Statement}
Show that the Metropolis-Hastings transition distribution $\mathcal{T}$ satisfies detailed balance with respect to the target distribution $p^*$. 

That is, show that:
\begin{align}
p^*( a) \mathcal{T}( b | a)  = p^*(b) \mathcal{T}( a | b)
\end{align}
for all possible $a \neq b$, where $a, b$ are any two distinct values of the random variable.
}

\subsection{2a: Solution}
TODO YOUR SOLUTION HERE



\officialdirections{
\subsection*{3a: Problem Statement}
(See diagram on 3a Instructions web page)

You start at Medford/Tufts station, and take 1000 steps. What is your probability distribution over ending this journey at each of the 7 stations? Report as a vector (use order of nodes in the diagram, small to large). Round to 3 decimal places.
}

\subsection{3a: Solution}
TODO YOUR SOLUTION HERE

[0.000 0.000 0.000 0.000 0.000 0.000 0.000]

\officialdirections{
\subsection*{3b: Problem Statement}
Is there a unique stationary distribution for this Markov chain? If so, explain why. If not, explain why not.
}


\subsection{3b: Solution}
TODO YOUR SOLUTION HERE


\end{document}
